\documentclass{beamer}
\usepackage[utf8]{inputenc}
\usepackage{default}
\usepackage{multirow} 
\usepackage{amsmath}
\usetheme{Berkeley}
\title{Redes Auto-Organizáveis: Mapas de Kohonen}
\author{Guilherme, Gustavo, Sean e Vinícius}
\institute{Universidade Estadual de Londrina}

\begin{document}

\frame{\titlepage}
\frame {
\frametitle{Sumário}
\tableofcontents
}

\section{Treinamento}


\begin{frame}{Treinamento}
	\begin{itemize}
	  \item Aprendizado competitivo e não-supervisionado.
	  \item Um padrão de entrada é apresentado a rede, um neurônio vencem e inicia a atualização dos pesos do neurônio vencedor e de seus vizinhos.
	  \item A taxa de aprendizado e o raio de vizinhança são decrementados durante o processo de aprendizagem.
	\end{itemize}
\end{frame}


\begin{frame}{Treinamento}

	\begin{itemize}
	  \item Atualização dos pesos do neurônio vencedor e de seus vizinhos

\[ w_{ji}(t+1) = \left\{ 
  \begin{array}{l l}
    w_{ji}(t) + \eta(t)(x_{i}- w_{ji}(t)) & \quad \text{se j $\in$ $\Lambda$(t)}\\
    w_{ji}(t) & \quad \text{caso contrário}
  \end{array} \right.\]

	\end{itemize}
\end{frame}


\begin{lstlisting}
	 Inicializar pesos e parametros;
	 repita
	    para todo padrao de treinamento faca
	       Definir neuronio vencedor;
	       Atualizar os pesos deste neuronio e de seus vizinhos;
	       se o numero do ciclo for multiplo de N entao
	          Entao reduzir a taxa de aprendizado;
	       fim-se
	    fim-para
	 ate que mapa de caracteristicas nao mudar
\end{lstlisting}


\section{Aplicações}

\begin{frame}{Aplicações}
	\begin{itemize}
	  \item Reconhecimento de padrões
	  \item Agrupamento de dados
	  \item Mapeamento da voz
	\end{itemize}
\end{frame}

\section{Titulo}

\begin{frame}{Titulo}
	\begin{itemize}
	  \item Bla
	\end{itemize}
\end{frame}

\begin{frame}{Resultados}
	\begin{itemize}
	  \item A implementação do modelo de Mapa de Kohonen, desenvolvida por Teuvo Kohonen em 1982.
	  \item Base de dados utilizada \textit{UCI Repository}.
	  \item Exemplo escolhido Car Evaluation.
	\end{itemize}
\end{frame}

\begin{frame}{Conjunto de treinamento}
	\begin{itemize}
	  \item Conjunto de exemplos adaptado do original. 
	  \item Conjunto de exemplos de três dimensões: Valor de compra, custo da manutenção e segurança.
	  \item 1728 instâncias.
	  \item Testes variando os atributos: quantidade de neurônios, taxa de aprendizado, raio da vizinhança e pesos iniciais.
	\end{itemize}
\end{frame}

\begin{frame}{Dúvidas ?}
	\begin{itemize}
	  \item https://github.com/QSF/ia-kohonen
	\end{itemize}
\end{frame}

\end{document}
