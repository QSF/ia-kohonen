\title{Redes Auto-Organizáveis: Mapas de Kohonen}
\author{}
\date{\today}

\documentclass[12pt]{article}
\usepackage{amsmath}

\begin{document}
\maketitle

\begin{abstract}
Este artigo apresenta uma implementação de um modelo de rede neural auto-organizável conhecida como mapas auto-organizáveis de Kohonen ou redes SOM (do inglês \textit{Self-Organizing Maps}).
\end{abstract}

\section{Introdução}\label{introducao}

\section{Redes SOM}\label{som}

\subsection{Treinamento}\label{treinamento}
A rede SOM utiliza um algoritmo de aprendizado competitivo e não-supervisionado. para o aprendizado da rede. Um padrão de entrada é apresentado a rede, um neurônio vencem e inicia a atualização dos pesos do neurônio vencedor e de seus vizinhos (até um raio de vizinhança). Isto repete para cada nova entrada e a taxa de aprendizado e o raio de vizinhança são decrementados durante o processo.[Braga]

Atualização dos pesos do neurônio vencedor e de seus vizinhos:

\[ w_{ji}(t+1) = \left\{ 
  \begin{array}{l l}
    w_{ji}(t) + \eta(t)(x_{i}- w_{ji}(t)) & \quad \text{se j $\in$ $\Lambda$(t)}\\
    w_{ji}(t) & \quad \text{caso contrário}
  \end{array} \right.\]

Onde $w_{ji}$ é o peso entre os neurônios i e j, $\eta(t)$ é a taxa de aprendizado e $\Lambda$ é a vizinhança.

\begin{verbatim}
1:  Inicializar pesos e parâmetros;
2:  repita
3:     para todo padrão de treinamento faça
4:        Definir neurônio vencedor;
5:        Atualizar os pesos deste neurônio e de seus vizinhos;
6:        se o número do ciclo for múltiplo de N então
7:           Então reduzir a taxa de aprendizado;
8:        fim-se
9:     fim-para
10: até que mapa de características não mudar
\end{verbatim}

\section{Métodos}\label{metodos}


\section{Resultados}\label{resultados}


\section{Conclusão}\label{conclusao}


\bibliographystyle{abbrv}
\bibliography{simple}

\end{document}